\documentclass[11.5pt, twoside, a4paper, titlepage]{report}
\usepackage{graphicx, amssymb, amsmath, mathtools, amsthm}
\usepackage{tikz}
\usetikzlibrary{automata, arrows}
\providecommand{\abs}[1]{\lvert#1\rvert}
\providecommand{\equ}[1]{\begin{equation*}}
\providecommand{\eequ}[1] {\end{equation*}}
\theoremstyle{definition}
\newtheorem{mydef}{Definition}[section]
\newtheorem{rem}[mydef]{Remark}
\newtheorem{note}[mydef]{Notation}
\newtheorem{eg}[mydef]{Example}
\theoremstyle{plain}
\newtheorem{lem}[mydef]{Lemma}
\newtheorem{thm}[mydef]{Theorem}
\newtheorem{cor}[mydef]{Corollary}



\begin{document}
\title{Representation of Quivers}
\author{Chanelle Lee \\Student ID: 200646370\\Supervisor: William Crawley-Boevey}
\date{\today}
\maketitle


\tableofcontents

\chapter{Introduction}

\chapter{Homological Algebra}
\section{Chain Complexes}

\mydef{A \emph{chain complex} $C_{\bullet}$ consists of a sequence of $\mathbb{R}$-modules $C_i$ ($i \in \mathbb{Z}$) and morphisms of the form,
\begin{equation*}
C: \qquad \dots \xrightarrow{} C_2 \xrightarrow{\delta_{2}} C_1 \xrightarrow{\delta_{1 }} C_0 \xrightarrow{\delta_0} C_{-1} \xrightarrow{\delta_{-1}} C_{-2} \xrightarrow{\delta_{-2}}\dots
\end{equation*}
such that $\delta_{n-1}\delta_{n}=0$ for all $n$, i.e. the composition of any two consecutive maps is zero. The maps $\delta_n$ are called the \emph{differentials} of $C$.}

\rem{It is convention that the map $\delta_n$ starts at $C_n$. }

\mydef{If $C$ is a chain complex then its \emph{homology} is defined to be,
\begin{equation*}
H_n(C)=\frac{Ker(\delta_n:C_n \rightarrow C_{n-1})}{Im(\delta_{n+1}:C_{n+1} \rightarrow C_n)} =\frac{Z_n(C)}{B_n(C)}.
\end{equation*}
This becomes an $\mathbb{R}$-module and, since $\delta^2$, it follows that $B_n(C)\subseteq Z_n(C)$.}

\eg{
 If we take a module $M$ then we can make a chain complex;
\begin{equation*}
C: \qquad \dots \xrightarrow{} 0 \xrightarrow{} M \xrightarrow{} 0 \xrightarrow{} \dots
\end{equation*}
where $M$ is at degree $n$. Then the homology will be:
\begin{equation*}
H_i(C)=\begin{cases}
\frac{Ker(M \rightarrow 0)}{Im(0 \rightarrow M)}=M & i=n\\
0 & i \neq n
\end{cases}
\end{equation*}
}

\eg{
If we have a field $K$ then we can create the following chain complex:
\begin{equation*}
C': \qquad \dots \xrightarrow{} 0 \xrightarrow{} K^2 \xrightarrow{\bigl(\begin{smallmatrix}1&3&0\\2&0&0\end{smallmatrix}\bigr)} K^3 \xrightarrow{(\begin{smallmatrix}0&0&1\end{smallmatrix})} K^1 \xrightarrow{} 0 \xrightarrow{}\dots
\end{equation*}
We can clearly see that the maps uphold the $\delta^2=0$ condition as,
\begin{equation*}
\begin{pmatrix}
1 & 3 & 0\\
2 & 0 & 0
\end{pmatrix}
\begin{pmatrix}0\\ 0\\ 1
\end{pmatrix}
=\begin{pmatrix}
0 & 0
\end{pmatrix}.
\end{equation*}
Then if we set $K^2$ at degree $0$ we can see that the homology becomes,
\equ
H_i(C')=
\begin{cases}

}

\mydef{
\begin{itemize}
\item The elements of $B_n(C)$ are called \emph{$n-$boundaries}.
\item The elements of $Z_n(C)$ are called \emph{$n-$cycles}.
\end{itemize}}

\rem{If $x\in Z_n(C)$ then its image in $H_n(C)$ is usually written as $[x]$. }

\mydef{A chain complex C is said to be:
\begin{itemize}
\item \emph{acyclic} if $H_n(C)=0$ for all $n$.
\item \emph{bounded above} if there exists some $n\in \mathbb{N}$, $C_k=0$ for all $k>n$.
\item \emph{bounded below} if for some $n \in \mathbb{N}$, $C_k=0$ for all $k<n$.
\item \emph{bounded} if it is bounded above and below.
\item \emph{non-negative}  if $C_n=0$ for all $n$.
\end{itemize}}

\mydef{A \emph{cochain complex} $C^{\bullet}$ consists of a sequence of $\mathbb{R}$-modules $C^i$ ($i \in \mathbb{Z}$) and morphisms of the form,
\begin{equation*}
\dots \xrightarrow C^{-2} \xrightarrow{\delta^{-2}} C^{-1} \xrightarrow{\delta^{-1 }} C^0 \xrightarrow{\delta^0} C^{1} \xrightarrow{\delta^{1}} C^{2} \xrightarrow{\delta^{2}} \dots
\end{equation*}
such that $\delta^{n-1}\delta^{n}=0$ for all $n$, i.e. the composition of any two consecutive maps is zero.







\end{document}
