\documentclass[11.5pt, twoside, a4paper, titlepage]{report}
\usepackage{graphicx, amssymb, amsmath, mathtools, amsthm}
\usepackage{tikz}
\usetikzlibrary{automata, arrows}
\providecommand{\abs}[1]{\lvert#1\rvert}
\theoremstyle{definition}
\newtheorem{mydef}{Definition}[section]
\newtheorem{rem}[mydef]{Remark}
\newtheorem{note}[mydef]{Notation}
\newtheorem{eg}[mydef]{Example}
\theoremstyle{plain}
\newtheorem{lem}[mydef]{Lemma}
\newtheorem{thm}[mydef]{Theorem}
\newtheorem{cor}[mydef]{Corollary}



\begin{document}
\title{Representation of Quivers}
\author{Chanelle Lee \\Student ID: 200646370\\Supervisor: William Crawley-Boevey}
\date{\today}
\maketitle


\tableofcontents

\chapter{Introduction}

\chapter{Homological Algebra}
\section{Chain Complexes}

\mydef{A \emph{chain complex} $C$ consists of a sequence of $\mathbb{R}$-modules $C_i$ and homomorphisms of the form,
\begin{equation*}
... \xrightarrow C_2 \xrightarrow{\delta_{2}} C_1 \xrightarrow{\delta_{1 }} C_0 \xrightarrow{\delta_0} C_{-1} \xrightarrow{\delta_{-1}} C_{-2} \xrightarrow ...
\end{equation*}
such that $\delta_{n-1}\delta_{n}=0$ for all $n$, i.e. the composition of any two consecutive maps is zero. The maps $\delta_n$ are called the \emph{differentials} of $C$.}

\rem{It is convention that the map $\delta_n$ starts at $C_n$. }

\mydef{If $C$ is a chain complex then its \emph{homology} is defined to be,
\begin{equation*}
H_n(C)=\frac{Ker(\delta_n:C_n \rightarrow C_{n-1})}{Im(\delta_{n+1}:C_{n+1} \rightarrow C_n)} =\frac{Z_n(C)}{B_n(C)}.
\end{equation*}







\end{document}
