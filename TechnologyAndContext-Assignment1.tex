\documentclass[11.5pt, twoside, a4paper]{article}

\begin{document}

\title{Technology and Context of Robotics and Autonomous Systems \\ Assignment 1: Report on Attendance}
\author{Chanelle Lee}
\date{October 2015}
\maketitle

\textbf{Assisted Living (Dr Praminda Caleb-Solly):} 

\textbf{Mobile Robotics (Dr Arthur Richards):} The research area of mobile robotics concerns the study of robots which have at least one form of unconstrained motion, i.e. a driverless car is essentially able to move along the ground in any direction for any distance, whilst a robot arm can only move a fixed radius from its base. Mobile robots can be found in a variety of environments, including on the ground, in the air, underwater and even in space, so whilst mobile robots share some common problems, many problems faced are specific to the robot's environment. Although mobile robotics is a fascinating area of research, it must be kept in mind that sometimes a mobile robot is not always the most cost-effective and practical solution. \\
\emph{Research challenge:} When a mobile robot moves it will drift slightly and this is a universal challenge across mobile robotics. How do we deal with the unknown amount of drift a robot might experience as it moves?
\end{document}